\documentclass[]{achemso}

\usepackage{amsmath}        % Equation editing using flags of \begin{align} and \end{align}
\usepackage{graphicx}       % Display figures using \includegraphics
\graphicspath{{figures/}}   % Location for figures relative to .tex file path
\usepackage{etoolbox}       % Make the bibliograph unjustified (to work around hbox errors)
\apptocmd{\thebibliography}{\raggedright}{}{}

%\usepackage{booktabs}       % Use table things like \toprule or \bottomrule
%\usepackage[table]{xcolor}  % Lets us use \rowcolors to make alternate table shading
%\usepackage{makecell}       % Allow the use of the \makecell command that lets linebreaks in table cells
%\usepackage{pdflscape}      % Enable the \begin{landscape} command

% chktex-file 36


%%%%%%%%%%%%%%%%%%%% Title/Abstract %%%%%%%%%%%%%%%%%%%%
\title{Uncertainty Benchmark (placeholder)}
\author{Kevin Tran}
\affiliation{Department of Chemical Engineering, Carnegie Mellon University, Pittsburgh, PA 15217}
\author{Willie Neiswanger}
\affiliation{Department of Machine Learning, Carnegie Mellon University, Pittsburgh, PA 15217}
\author{Junwoong Yoon}
\affiliation{Department of Chemical Engineering, Carnegie Mellon University, Pittsburgh, PA 15217}
\author{Eric Xing}
\affiliation{Department of Machine Learning, Carnegie Mellon University, Pittsburgh, PA 15217}
\author{Zachary W. Ulissi}
\affiliation{Department of Chemical Engineering, Carnegie Mellon University, Pittsburgh, PA 15217}
\email{zulissi@andrew.cmu.edu}

\begin{document}

%\setlength{\fboxrule}{0 pt}
%\begin{tocentry}
%    \includegraphics[width=\textwidth]{TOC/TOC.pdf}
%    This perspective discusses three common tools used in informatics research:
%    databases, surrogate modeling, and workflow managers. Although these tools are not
%    new, they are relatively new in the field of surface science and catalysis. We
%    discuss how these tools can augment and accelerate surface science research, and we
%    provide examples from both literature and our own work. We also provide our
%    perspective on when to use these tools and some best practices to follow when
%    creating them.
%\end{tocentry}

\begin{abstract}
    Abstract here.
\end{abstract}


%%%%%%%%%%%%%%%%%%%% Introduction %%%%%%%%%%%%%%%%%%%%

\section{Introduction}

Intro here. Proxy citation here.\cite{Tran2018}


%%%%%%%%%%%%%%%%%%%% Methods %%%%%%%%%%%%%%%%%%%%

\section{Methods}

Methods here.

%%%%%%%%%%%%%%%%%%%% Results %%%%%%%%%%%%%%%%%%%%

\section{Results}

%\subsection{Foo}


%%%%%%%%%%%%%%%%%%%% Conclusions %%%%%%%%%%%%%%%%%%%%

\section{Conclusions}

Conclusions here


%%%%%%%%%%%%%%%%%%%% Misc %%%%%%%%%%%%%%%%%%%%

\section*{Code availability} Visit \texttt{https://github.com/ulissigroup/uncertainty\_benchmarking} for the code used to create the results discussed in this paper.
The code dependencies are listed inside the repository.

\section*{Author information} Corresponding author email:  zulissi@andrew.cmu.edu.
The authors declare no competing financial interest.

\section*{Acknowledgements} This research used resources of the National Energy Research Scientific Computing Center, a DOE Office of Science User Facility supported by the Office of Science of the U.S. Department of Energy under Contract No. DE-AC02-05CH11231. % chktex 8
% TODO add credit to NESAP
% TODO add credit to Willie's cluster


%%%%%%%%%%%%%%%%%%%% Bibliography %%%%%%%%%%%%%%%%%%%%

\clearpage
\bibliography{uncertainty_benchmarking}

\end{document}
